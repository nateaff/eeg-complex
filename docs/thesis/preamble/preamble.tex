% preamble.tex, to be used with thesis.tex
% This contains the TeX definitions for layout, style, etc., as well as the first few pages of your thesis: title page, copyright page, approval page, abstract, acknowledgments, tables of contents, tables, and figures.
% The layout commands should give the correct margins according to the graduate division's guidelines

%%%%% TeX class and packages

\documentclass[12pt,oneside]{sfsuthesis}

% ==================================
% additional packages

\usepackage{amsthm,amsmath,amssymb,amsfonts,latexsym,graphicx,enumerate,setspace,verbatim,tocloft,rotating,color}

% =====================================
\usepackage[ruled,vlined]{algorithm2e}
\usepackage{hhline}
% \usepackage{algorithm}
\usepackage{algorithmicx}
% \usepackage{algpseudocode}
\usepackage{subcaption}
\usepackage{caption}
\usepackage{subfig}
\usepackage[export]{adjustbox}
\usepackage{float}

% \graphicspath{{./figs/}}

% ====================================

%\usepackage{color}                    % For creating colored text and background
%\usepackage{hyperref}                 % For creating hyperlinks in cross references
% other possibly useful packages: textcomp,mathrsfs,amscd,epsfig,euscript,cancel

%%%%% Layout
% These numbers might depend on your printer. Check the margins and compare them to the Graduate Division's
% guidelines. If there's something off, try playing with the numbers...
%
% For chapters:
%     Must have a minimum of 1.5in margin on left and 1in on all other sides.  Where there are page numbers
%     (whether on top or bottom), must have one additional inch between the page number and the text, for a
%     total of 2in between the edge of the paper and the text.
% For frontmatter pages:
%     The same margin numbers generally work, except for the Title Page, so you will notice that we use
%     some numbers for \textheight and \footskip right here, and then change them below, right after
%     generating the Title Page.

\hoffset=.5in 
\oddsidemargin=0in   % = 1in because LaTeX adds 1in
\evensidemargin=0in  % = 1in because LaTeX adds 1in
\topmargin=0in       % = 1in because LaTeX adds 1in
\headheight=0in
\headsep=1in         % Distance from top of pagenum (for page numbers at top-right corner of page) to text
\footskip=1.2in      % Distance from bottom of text to the page number (for page number at bottom of page)
\textwidth=5.9in     % Should be 6in, but use 5.9in to be conservative
\textheight=8.0in    % Best for Title Page (will change after the Title Page)

\pagestyle{plain}

\doublespacing

%%%%% Style of theorems, definitions, examples, equations, etc.

\theoremstyle{plain} % Heading is bold, text italic.
\newtheorem{theorem}{Theorem}[chapter]
\newtheorem{lemma}[theorem]{Lemma}
\newtheorem{proposition}[theorem]{Proposition}
\newtheorem{corollary}[theorem]{Corollary}
\newtheorem{conjecture}{Conjecture}[chapter]

\theoremstyle{definition}  % Heading is bold, text is roman
\newtheorem{definition}{Definition}[chapter]
\newtheorem{example}{Example}[chapter]

\theoremstyle{remark}  % Heading is italic, text is roman
\newtheorem*{remark}{Remark}
\newtheorem*{note}{Note}
\newtheorem{claim}{Claim}[chapter]

%%%%% Appendix style

\renewcommand\appendix[1]{
\chapter*{#1}
\addcontentsline{toc}{chapter}{#1}
}

%%%%% Title page

\begin{document}

\pagenumbering{roman}
\thispagestyle{empty}

\[ \]
\vspace{-1.9in}

\begin{center}
{\mytitle}

\vspace{1.4in}

\singlespace{A thesis presented to the faculty of\\
San Francisco State University\\
In partial fulfilment of\\
The Requirements for\\ The Degree}

\vspace{.5in}

\singlespace{\mydegree \\ In\\ \myfield}

\vspace{3.1in}

{by \\[12pt] 
\myname \\[12pt]
San Francisco, California\\[12pt]
\thismonth
\thisyear}
\end{center}

% \newpage
% \textheight=7.1in    % For all pages after the Title Page -- try making this number smaller (7.0 or 6.9) if the bottom margins are too small
% \footskip=1.1in      % Distance from bottom of text to the page number (for page number at bottom of page)
% \thispagestyle{empty}

% $\mbox{}$
% \vspace{3in}
% \begin{center}
% \singlespace{
% Copyright by\\ 
% \myname \\
% \thisayear
% }
% \end{center}

\newpage
\thispagestyle{empty}
\[ \]
\vspace{-1.8in}
\begin{center}
{CERTIFICATION OF APPROVAL}
\end{center}
\vspace{.6in}
\begin{quote}
I certify that I have read {\it \mytitle} by \myname and that in my opinion this work meets the criteria for approving a thesis submitted in partial fulfillment of the requirements
for the degree: \mydegree in \myfield at San Francisco State University.
\end{quote}

\vspace{0.75in}

\hspace*{\fill}\parbox{3.5in}{
\singlespace{

\hrule{\hspace{3.5in}} \\ 
Dr. Alexandra Piryatinska\\
Associate Professor of \myfield

\vspace{1in}

\hrule{\hspace{3.5in}} \\
Dr. Tao He\\
Assistant Professor of \myfield 

\vspace{1in}

\hrule{\hspace{3.5in}} \\
Dr. Chun Kit Lai\\
Assistant Professor of \myfield 

}
}

\newpage
\thispagestyle{empty}
\[ \]
\vspace{-1.8in}
\begin{center}
{\mytitle} \\

\vspace{.5in}

\singlespace{
\myname \\
San Francisco State University \\ 
\thisyear \\
}

\end{center}

\vspace{.3in}

\doublespacing{\noindent
The $\varepsilon$-complexity of a continuous function is a measure of the amount of information needed to reconstruct a function with an absolute error not larger than $\varepsilon$. For
H\"older class functions, $\varepsilon$-complexity is characterized by a pair of real numbers, the complexity 
coefficients.
The complexity coefficients have been shown to be useful features for the segmentation and classification of time series. 
In this work, we extend the set of
approximation methods used in estimating the complexity 
coefficients. The performance of these approximation methods is 
tested on a number of simulated time series.
% A set of simulations is used to test 
% the performance of this enlarged set of methods.
In addition, we test the conjecture that, for a given generating mechanism, the mean of the complexity coefficients is constant.
For our set of simulations, we find that the mean of the estimated complexity coefficients is constant on a constant H\"{o}lder class of functions. Finally, we apply the $\varepsilon$-complexity coefficients to the prediction of seizures in epileptic mice. 
% In particular, the $\varepsilon-$complexity coefficients are used to segment the EEG signal.   
%  Change points in the complexity coefficients are uased to 
%  segement the EEG signal and the average feature values on these
%  segments serve as predictors of a seizure outcome.
 We use this technique to identify which EEG signal preceded a seizure with over 80\% accuracy. 

}

\vspace*{\fill}

\hspace*{\fill}

\noindent
I certify that the Abstract is a correct representation of the content of this thesis.

\vspace{.6in} 

\hrule{\hspace{3.75in}} \\[-10pt]
Chair, Thesis Committee 
\hspace{2.5in}
Date

\newpage
\[ \]
\vspace{-1.8in}
\begin{center}{ACKNOWLEDGMENTS}\end{center}

\vspace{.3in}
\begin{quote}
\noindent
% I thank my advisor Dr. Alexandra Piryatinska her for her guidance 
% My two committee members Dr. Tao He and Dr. Chun-Kit Lai I thank 
% for their patience.

% And bicycle helmets off to
%  all my fellow students with whom I discussed math and problems of
 % other sorts over the last few years.
\end{quote}

\renewcommand{\contentsname}{\vspace{-1.8in} \begin{center} \normalsize \rm TABLE OF CONTENTS \end{center}}
\renewcommand{\listfigurename}{\vspace{-1.8in} \begin{center} \normalsize \rm LIST OF FIGURES \end{center}}
\renewcommand{\listtablename}{\vspace{-1.8in} \begin{center} \normalsize \rm LIST OF TABLES \end{center}}
\renewcommand{\cftchapfont}{\normalfont}
\renewcommand{\cftchappagefont}{\normalfont}
\renewcommand{\cftchapleader}{\cftdotfill{\cftdotsep}} % formatting commands for table of contents
\renewcommand{\cftsecfont}{\normalfont}
\renewcommand{\cftsecpagefont}{\normalfont}
\renewcommand{\cftsecleader}{\cftdotfill{\cftdotsep}}

\newpage \tableofcontents 
\newpage Table \hfill Page \listoftables % comment out if you don't use tables
\newpage Figure \hfill Page \listoffigures % comment out if you don't use figures

\newpage
\pagestyle{myheadings}
\pagenumbering{arabic} 
\setcounter{page}{1}
